\begin{abstract} 
%Increasing concerns on user privacy, the growing pervasiveness of sensing and distributed data generation, and the rise in computing demand at the edge of the network are calling for a shift in paradigm towards fully decentralized machine learning approaches, which guarantee better scalability and privacy protection trhan traditional, centralized architectures. In this context, a special role is played by Gossip Learning (GL), a peer-to-peer machine learning protocol based on direct, opportunistic exchange of models among nodes via wireless D2D communications, and on collaborative model training, which has recently proven to scale efficiently to large number of nodes.
%employs a client-server architecture with each node acting as client and server, possibly at the same time. GL is not based on a central server and is therefore highly scalable. Each node exchanges local model instance with neighbors, Neighbors train received models based on their local dataset and send partially trained model back,with no raw data exposure, so that user privacy is preserved.
%Nodes train their own model instance based on their local dataset, and exchange model instances with neighbors, with no raw data exposure, so that user privacy is preserved.\\
Gossip Learning (GL) is a peer-to-peer machine learning protocol based on direct, opportunistic exchange of models among nodes via wireless D2D communications, and on collaborative model training, which has recently proven to scale efficiently to large numbers of nodes, and to offer better privacy guarantees than traditional centralized learning architectures.
Existing approaches to GL are however limited to scenarios in which nodes are static, or in which the node connectivity graph is fully connected, and they are fragile to node churn as well as to any change in network configuration. 
%It is thus so far unclear how to apply them in realistic, dynamic setups where nodes join and leave the network, such as in vehicular ad-hoc networks (VANETs).\\
To overcome this limitation, we present a new decentralized architecture for GL suitable for setups with dynamic nodes,  which benefits from node mobility instead of being hampered by it. In our approach, nodes improve their personalized model instance by sharing it with neighbors, and by weighting neighbors' contributions according to an estimate of their marginal utility. % Hence, as many local consensus models are developed as the number of nodes.\\
We apply our GL algorithm to short-term vehicular trajectory estimation in realistic urban scenarios. %Specifically, we assume that data is unevenly distributed across vehicles, and that \com{no vehicle obtains a representative sample of the model instances available in the overall population.}
%(ii) each vehicle only collects a tiny fraction of the data produced by the whole , and (iii) no vehicle obtains a representative sample of the model instances available in the overall population. 
We propose a new strategy for the estimation of the neighbors' instances marginal utility, which yields satisfactory trajectory estimation accuracy for nodes with long enough sojourn times.

%Then we define three  practical gossip learning algorithms called DFed Avg, DFed Pow, and DFed Best. We applied an LSTM model on a dynamic time series dataset while connectivity graph amongst nodes is not static.\\
%Numerical evaluations show that these approaches perform well in dynamic scenarios, with high accuracy () and low loss (). %particularly when vehicles spend an adequately amount of time (min 20 min).
 \end{abstract}