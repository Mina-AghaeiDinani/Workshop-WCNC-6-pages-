\vspace{-0.15in}
\section{Conclusions and future work}
\label{sec:Conclusion}
The preliminary results presented in this paper suggest that the performance of nodes with too small/poor datasets, or short sojourn times might be improved by having nodes with better models spread them opportunistically, and let other nodes use such received models as starting point for the GL algorithm. We thus plan on expanding the approach discussed in this paper with other forms of model exchanges among vehicles, which can better enable vehicles with short sojourn times and/or small datasets to contribute significatively to the model sharing scheme. Moreover, we plan on performing a thorough assessment of our algorithms on a variety of other vehicular scenarios, and to characterize their convergence properties.

%\com{@Mina: please review all cited works, making style uniform across all of them (not all have same level of detail), and adding capitalization wherever needed (you need to close the capitalized word within curly braces).}
%\vspace{-7pt}
% Currently, we are using real mobility scenarios in order to compare the results with our model. We would like to use this approach as support to localization system as GPS. Indeed, a vehicle could estimate its position though the position of another vehicle. 
%\vspace{-3pt}